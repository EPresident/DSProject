\documentclass[12pt]{article}
\usepackage[latin1]{inputenc}
\usepackage[italian]{babel}
\usepackage{microtype} % tipografia
\usepackage[bookmarks=true,bookmarksopen=true,pdfhighlight=/I,pdfpagemode=UseOutlines]{hyperref} % hyper-links, questo va per ultimo


\makeindex

\begin{document}
\title{Relazione del progetto di \\Sistemi Distribuiti}
\author{Elia Calligaris e Paolo Snidaro}
\date{\textit{bozza} - \today}
\maketitle
\tableofcontents

\section{Introduzione}\label{sec:intro}

Il problema proposto � quello di implementare un protocollo di transazioni distribuite, tramite il quale dei client possono prenotare concorrentemente delle risorse disposte su pi� server.

\paragraph{Il pretesto} � quello di gestire un sistema di prenotazione di posti su voli aerei. Idealmente un'agenzia di viaggio (client) deve essere in grado di prenotare via internet una serie di posti su pi� voli di diverse compagnie. I server delle compagnie aeree mantengono la lista e lo stato dei voli e relativi posti. La prenotazione ha successo se e solo se \textit{tutti} i posti richiesti vengono prenotati. 

Naturalmente si vuole evitare che le richieste di prenotazione interferiscono tra di loro (e soprattutto evitare che lo stesso posto venga prenotato pi� volte). Nel caso in cui venga richiesto un posto che risulta occupato, la compagnia aerea dovrebbe proporre un posto alternativo (se possibile).

\paragraph{Una soluzione na\"ive} � quella di creare una transazione per ogni posto da prenotare, abortendo nel caso in cui una fallisca; tuttavia si vuole evitare l'annullamento di prenotazioni gi� effettuate.

\paragraph{Considerazioni} Si suppone che il sistema venga operato da umani (gli impiegati delle agenzie di viaggio), perci� � lecito ritenere che le probabilit� di interferenza tra prenotazioni sia basso.

\section{Analisi}\label{sec:analisi}

\subsection{Requisiti funzionali}
\begin{itemize}
	\item Ogni agenzia deve poter richiedere i voli e i posti disponibili alle compagnie.
	\item Ogni agenzia deve essere in grado di prenotare atomicamente una serie di posti su voli potenzialmente diversi. A fronte della lista di posti richiesti, viene richiesta la conferma o meno della prenotazione, eventualmente con delle proposte alternative a posti richiesti ma occupati.
	\item Qualora un'agenzia non riesca ad ottenere uno o pi� posti richiesti, l'intera prenotazione deve essere annullata.
	\item Qualora un client (agenzia) subisca guasti, la transazione va abortita.
	\item Ci devono essere almeno cinque compagnie aeree.
	\item Ogni server di ogni compagnia aerea mantiene una lista di voli, i quali hanno una serie di posti che possono essere liberi o prenotati.
	\item Un posto in aereo non deve mai venire prenotato pi� volte.
	\item Una volta confermata una prenotazione per un posto aereo, questa non deve essere pi� annullata.
\end{itemize}

\subsection{Requisiti non funzionali}
\begin{itemize}
	\item Fault tolerance: i server delle compagnie aeree devono cercare di mascherare i guasti.
	\item Trasparenza rispetto alla replicazione.
	\item Alt� disponibilit�, almeno $96\%$ (circa un giorno al mese di downtime).
\end{itemize}

\end{document}














